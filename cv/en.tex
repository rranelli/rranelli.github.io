\documentclass{resume} % Use the custom resume.cls style
\usepackage[utf8]{inputenc}
\usepackage[left=0.75in,top=0.6in,right=0.75in,bottom=0.6in]{geometry} % Document margins

\name{Renan Ranelli} \address{(11)~988xy~$\cdot$~90wz \\ renanranelli@gmail.com}
\address{github.com/rranelli \\ rranelli.com} \address{~São
  Paulo~$\cdot$~Sp~$\cdot$~Brasil}

% ---

\begin{document}
\begin{rSection}{Professional Objective}
  I'm looking for a position as a software developer in a company with an
  open and informal environment. This company must aim for technical excellence,
  continuous improvement and no fear of error.

  I \underline{love} to study computing and I am also a highly technical person.
  Hence, I'm looking for a position that enables one to face highly complex
  technical challenges.
\end{rSection}

\begin{rSection}{Education}
  {\bf University of Campinas, Unicamp} \hfill {\em January 2014 - December 2014} \\
  Specialization: Software Engineering \\
  GPA higher than 85\%

  {\bf University of Campinas, Unicamp} \hfill {\em January 2012 - Incomplete} \\
  Master's degree: Chemical Systems Engineering \\
  While an undergraduate, I've concluded \underline{all} the classes necessary by
  the master's program. I have not completed the program due to my career change
  to software engineering.
  GPA: $A^-$

  {\bf University of Campinas, Unicamp} \hfill {\em February 2008 - December 2012} \\
  Bachelor's degree: Chemical Engineering, with honors \\
  GPA higher than 80\%
\end{rSection}

\begin{rSection}{Professional Experience}
  \begin{rSubsection}{Locaweb}{June 2014 - Today}{Software Engineer}{São
      Paulo, Brazil}

  \item Na Locaweb, faço parte do time de Hospedagem e plataforma (PaaS).
    Atualmente, trabalho na equipe da {\em nova plataforma de hospedagem} que
    possui o objetivo de \underline{reescrever} completamente todo o sistema de
    provisionamento de hospedagem da empresa, com o objetivo de aposentar o
    sistema legado de provisionamento (que já opera a mais de 13 anos). A
    Locaweb é a líder brasileira no mercado de hosting, possuindo mais de
    250.000 clientes.

  \item Trabalho em um ambiente bastante heterogêneo (Windows, Linux, C\#, Ruby,
    PHP, Perl, ASP) buscando torná-lo homogêneo e orientado a serviços. Sou
    responsável por todas as etapas da frente de desenvolvimento {\em core} do
    projeto. Nesta equipe trabalhamos com um {\em mix} de metodologias ágeis
    (scrum, kanban, XP).

  \item Trabalhamos com {\em TDD}, {\em pair programming}, integração contínua,
    {\em GIT}, fazemos {\em code-review/pull-requests}, e várias outras práticas
    de desenvolvimento ágil.

  \item Atuo primariamente com tecnologias Ruby-based, (e.g. Ruby on Rails,
    Sinatra, Grape, etc.) para desenvolvimento do {\em backend} de sistemas e
    apis web.
  \end{rSubsection}

  \begin{rSubsection}{Chemtech Software \& Engineering Services}{January 2012 -
      April 2014}{System's Engineer}{São Paulo, Brazil}

  \item Após ter obtido a terceira colocação na ``Maratona Nacional Chemtech de
    Engenharia 2011'' fui convidado a fazer parte do time de engenharia da
    Chemtech.

  \item Atuei no setor de {\em automação de processos} e {\em TI Industrial} no
    desenvolvimento e manutenção de sistemas críticos de apoio à operação
    industrial. Utilizei majoritariamente tecnologias do stack Microsoft
    (Windows Server, Sql Server, C\#, F\#, ASP.Net MVC).

  \item Fui responsável pela manutenção de um dos maiores sistemas de controle
    de armazém frigorífico automatizado do Brasil. Tive experiência com vários
    níveis de abstração, desde a comunicação direta com as máquinas em protocolo
    binário proprietário à orquestração de regras complexas de logística e
    otimização de inventário.

  \item Aproveitando a minha formação em engenharia, fui o pivô na consolidação
    de um grupo de automação de projetos, responsável por desenvolver soluções
    internas para facilitar e automatizar o trabalho das disciplinas de
    engenharia. Obtive recursos humanos e financeiros para estabelecer o grupo,
    e desenvolvemos um sistema de gestão de informações de projeto inédito na
    empresa.
  \end{rSubsection}
\end{rSection}

\begin{rSection}{Technical Knowledge}
  Most of my professional experience is in the Ruby/Linux and .Net/Windows
  ecosystems. I have solid knowledge in object oriented design and functional
  programming.

  I'm an Emacs power-user, and I have contributed to many open-source
  projects in Emacs-Lisp

  I have also completed many courses in the EDx and Coursera platforms, all
  100\% in English. Some of the courses I took were:

  \begin{tabular}{ @{} >{\bfseries} l @{\hspace{5ex}} l }
    $\cdot$ Programming languages & Coursera/University Of Washington \\
    $\cdot$ Software as a service & EDx/UC Berkley \\
    $\cdot$ Artificial intelligence & EDx/UC Berkley \\
    $\cdot$ Functional Programming (Haskell) & EDx/Université catholique de Louvain \\
    $\cdot$ Functional progamming principles in Scala & Coursera/École Polytechnique
                                                        Fédérale de Laussanne \\
    $\cdot$ Introduction to Databases & Coursera/Stanford \\
    $\cdot$ Computer Networks & Coursera/University Of Washington \\
    $\cdot$ Web Application Architectures & Coursera/University of New Mexico
  \end{tabular}

  I have deep understanding of object orientation and I have been studying
  functional programming quite a lot. For example, I have coded an object system
  similar to Ruby's in functionality. Also, I wrote a Scheme interpreter in
  Haskell.
\end{rSection}

\begin{rSection}{Languages}
  \begin{tabular}{ @{} >{\bfseries}l @{\hspace{6ex}} l }
    English & Fluent \\
    French & Intermediary
  \end{tabular}
\end{rSection}

\begin{rSection}{Complementary Academic Experience}
  {\bf Integrated Graduate Program (PIF) Unicamp} \hfill {\em January 2012 -
    December 2012} \\
  O programa de formação integrada permite que o aluno de graduação com
  destacado desempenho curse no ultimo ano de graduação disciplinas do programa
  de mestrado, visando facilitar a continuidade dos estudos. Cursei mais de 300
  horas em disciplinas do mestrado em engenharia de sistemas químicos. Entre
  elas, destaco:

  \begin{tabular}{ @{} l @{\hspace{6ex}} l }
    $\cdot$ Automation of chemical systems & 36 hours \\
    $\cdot$ Artificial intelligence applied to process control  & 30 hours \\
    $\cdot$ Computational methods applied to engineering & 60 hours \\
    $\cdot$ Linear control (electrical engineering's master degree course) & 60 hours \\
  \end{tabular}

  {\bf Undergraduate Research} \hfill {\em March 2009 - November 2012} \\
  \/Trabalhei com algoritmos de controle avançado para processos químicos.
  Integrei um grupo de pesquisa que realiza o desenvolvimento de protótipos
  (montagem, especificação, construção, configuração) para aplicação de
  algoritmos de controle avançado e validação de simulações computacionais.
  Nesse período, tive vasto contato com computação científica e desenvolvi
  bibliotecas em Fortran95, Matlab e Scilab. Durante todo o período tive bolsas
  de estudo da Fapesp e CNPq.
\end{rSection}
\end{document}
