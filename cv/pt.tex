\documentclass{resume} % Use the custom resume.cls style
\usepackage[utf8]{inputenc}
\usepackage[left=0.75in,top=0.6in,right=0.75in,bottom=0.6in]{geometry} % Document margins

\name{Renan Ranelli} \address{(11)~9881x~$\cdot$~90yz \\ renanranelli@gmail.com}
\address{github.com/rranelli \\ rranelli.com} \address{~São
  Paulo~$\cdot$~Sp~$\cdot$~Brasil}

% ---

\begin{document}
\begin{rSection}{Objetivo Profissional}
  Busco um cargo na área de desenvolvimento de software em uma empresa que
  busque um ambiente aberto e informal, que estimule a excelência técnica,
  melhoria contínua e que não tenha medo de errar.

  Sou \underline{apaixonado} por computação e por estudar, e busco um ambiente
  que valorize o desenvolvimento técnico dos colaboradores.
\end{rSection}

\begin{rSection}{Educação}
  {\bf Universidade Estadual de Campinas, Unicamp} \hfill {\em Janeiro 2014 - Dezembro 2014} \\
  Especialização: Engenharia de Software \\
  Média geral superior a 85\%

  {\bf Universidade Estadual de Campinas, Unicamp} \hfill {\em janeiro 2012 - Incompleto} \\
  Mestrado: engenharia de sistemas químicos \\
  Ainda durante a graduação conclui \underline{todos} os créditos requeridos
  pelo programa. \\
  Média geral: $A^-$

  {\bf Universidade Estadual de Campinas, Unicamp} \hfill {\em Fevereiro 2008 - Dezembro 2012} \\
  Graduação: Engenharia Química \\
  Média geral superior a 80\%
\end{rSection}

\begin{rSection}{Experiência Profissional}
  \begin{rSubsection}{Locaweb}{Junho 2014 - Atual}{Engenheiro de Software}{São
      Paulo, SP}

  \item Na Locaweb, faço parte do time de Hospedagem e plataforma (PaaS).
    Atualmente, trabalho na equipe da {\em nova plataforma de hospedagem} que
    possui o objetivo de \underline{reescrever} completamente todo o sistema de
    provisionamento de hospedagem da empresa, com o objetivo de aposentar o
    sistema legado de provisionamento (que já opera a mais de 13 anos). A
    Locaweb é a líder brasileira no mercado de hosting, possuindo mais de
    250.000 clientes.

  \item Trabalho em um ambiente bastante heterogêneo (Windows, Linux, C\#, Ruby,
    PHP, Perl, ASP) buscando torná-lo homogêneo e orientado a serviços. Sou
    responsável por todas as etapas da frente de desenvolvimento {\em core} do
    projeto. Nesta equipe trabalhamos com um {\em mix} de metodologias ágeis
    (scrum, kanban, XP).

  \item Trabalhamos com {\em TDD}, {\em pair programming}, integração contínua,
    {\em GIT}, fazemos {\em code-review/pull-requests}, e várias outras práticas
    de desenvolvimento ágil.

  \item Atuo primariamente com tecnologias Ruby-based, (e.g. Ruby on Rails,
    Sinatra, Grape, etc.) para desenvolvimento do {\em backend} de sistemas e
    apis web.
  \end{rSubsection}

  \begin{rSubsection}{Chemtech Serviços de Engenharia e Software}{Janeiro 2012 -
      Abril 2014}{Engenheiro de Sistemas}{São Paulo, SP}

  \item Após ter obtido a terceira colocação na ``Maratona Nacional Chemtech de
    Engenharia 2011'' fui convidado a fazer parte do time de engenharia da
    Chemtech.

  \item Atuei no setor de {\em automação de processos} e {\em TI Industrial} no
    desenvolvimento e manutenção de sistemas críticos de apoio à operação
    industrial. Utilizei majoritariamente tecnologias do stack Microsoft
    (Windows Server, Sql Server, C\#, F\#, ASP.Net MVC).

  \item Fui responsável pela manutenção de um dos maiores sistemas de controle
    de armazém frigorífico automatizado do Brasil. Tive experiência com vários
    níveis de abstração, desde a comunicação direta com as máquinas em protocolo
    binário proprietário à orquestração de regras complexas de logística e
    otimização de inventário.

  \item Aproveitando a minha formação em engenharia, fui o pivô na consolidação
    de um grupo de automação de projetos, responsável por desenvolver soluções
    internas para facilitar e automatizar o trabalho das disciplinas de
    engenharia. Obtive recursos humanos e financeiros para estabelecer o grupo,
    e desenvolvemos um sistema de gestão de informações de projeto inédito na
    empresa.
  \end{rSubsection}
\end{rSection}

\begin{rSection}{Conhecimentos Técnicos}
  Minha experiência profissional concentra-se principalmente nos ecossistemas
  Ruby/Linux e .Net/Windows. Tenho conhecimentos sólidos em design orientado a
  objetos.

  Sou um {\em power-user} do Emacs, e tenho contribuído para vários projetos
  {\em opensource} em Emacs-Lisp.

  Completei vários cursos nas plataformas EDx e Coursera, todos 100\% em língua
  inglesa, dentre os quais destaco:

  \begin{tabular}{ @{} >{\bfseries} l @{\hspace{5ex}} l }
    $\cdot$ Programming languages & Coursera/University Of Washington \\
    $\cdot$ Software as a service & EDx/UC Berkley \\
    $\cdot$ Artificial intelligence & EDx/UC Berkley \\
    $\cdot$ Functional Programming (Haskell) & EDx/Université catholique de Louvain \\
    $\cdot$ Functional progamming principles in Scala & Coursera/École Polytechnique
                                                        Fédérale de Laussanne \\
    $\cdot$ Introduction to Databases & Coursera/Stanford \\
    $\cdot$ Computer Networks & Coursera/University Of Washington \\
    $\cdot$ Web Application Architectures & Coursera/University of New Mexico
  \end{tabular}

  Domino programação orientada a objetos e atualmente venho estudando bastante a
  respeito de programação funcional. Aliás, já escrevi um sistema de objetos
  similar ao do Ruby em Clojure, e também um interpretador de Lisp (Scheme) em
  Haskell.
\end{rSection}

\begin{rSection}{Idiomas}
  \begin{tabular}{ @{} >{\bfseries}l @{\hspace{6ex}} l }
    Inglês & Fluente \\
    Francês & Intermediário
  \end{tabular}
\end{rSection}

\begin{rSection}{Formação Acadêmica Complementar}
  {\bf Programa de formação integrada (PIF) Unicamp} \hfill {\em Janeiro 2012 -
    Dezembro 2012} \\
  O programa de formação integrada permite que o aluno de graduação com
  destacado desempenho curse no ultimo ano de graduação disciplinas do programa
  de mestrado, visando facilitar a continuidade dos estudos. Cursei mais de 300
  horas em disciplinas do mestrado em engenharia de sistemas químicos. Entre
  elas, destaco:

  \begin{tabular}{ @{} l @{\hspace{6ex}} l }
    $\cdot$ Automação de sistemas químicos  & 36 horas \\
    $\cdot$ Inteligência artificial aplicada ao controle avançado  & 30 horas \\
    $\cdot$ Métodos computacionais aplicados à engenharia & 60 horas \\
    $\cdot$ Sistemas \& controle linear (curso do mestrado em eng. elétrica) & 60 horas \\
  \end{tabular}

  {\bf Iniciação Científica} \hfill {\em Março 2009 - Novembro 2012} \\
  \/Trabalhei com algoritmos de controle avançado para processos químicos.
  Integrei um grupo de pesquisa que realiza o desenvolvimento de protótipos
  (montagem, especificação, construção, configuração) para aplicação de
  algoritmos de controle avançado e validação de simulações computacionais.
  Nesse período, tive vasto contato com computação científica e desenvolvi
  bibliotecas em Fortran95, Matlab e Scilab. Durante todo o período tive bolsas
  de estudo da Fapesp e CNPq.
\end{rSection}
\end{document}
